
\documentclass[a4paper]{article}

\usepackage[english]{babel}
\usepackage[utf8x]{inputenc}
\usepackage{amsmath}
\usepackage{graphicx}
\usepackage[colorinlistoftodos]{todonotes}

\title{Quatrième rapport d'activité : lundi 2/05/2016 au lundi 9/05/2016 }
\author{Guillaume Maitrot}

\begin{document}
\maketitle

\begin{center}
\centering
\title{Consommation énergétique des systèmes embarqués}
\end{center}

\section{Contexte du stage}

 \subsection{Le travail réalisé depuis le commencement du stage}
 \begin{enumerate}
\item {Prise en main de la plateforme : compilation de code C pour la carte STM32, découverte de FreeRTOS}
\item {Programmation des périphériques de la carte : port série,
convertisseur Analogique-Numérique, carte mémoire flash}
\item {Développement des traitements nécessaires à l'application de
traitement du signal : acquisition de signal via le convertisseur,
traitement par une FFT, stockage du résultat de la FFT sur la carte
flash}
\end{enumerate}

 \subsection{Prochaine tâche}
    \paragraph{Programmation multi-threadée de l'ensemble sous FreeRTOS.}
    
\subsection{Mes tâches à accomplir}
\begin{enumerate}
\item { Programmation multi-threadée de l'ensemble sous FreeRTOS}
\item {Mesure de la consommation de l'application à l'aide l'Agilent
N6705}
\item {Expérimentation sur les possibilité de variation de fréquence sur la
STM32 concernant le processeur et les périphériques}
\item {Etude des variations de la consommation selon plusieurs paramètres :
fréquence des tâches, fréquence du processeur, fréquence des
périphériques}
\end{enumerate}

 \subsection{Le travail réalisé pendant la semaine}
 \begin{enumerate}
\item {J'ai finalisé les derniers bouts de code non "propre" pour pouvoir déposer sur git, mon travail effectué.Mais aussi de finir le portage sur freeRTOS des tâches de l'adc, la fft et l'écriture dans la carte mémoire flash.Puis de l'installation d'un génerateur de fonctions (GBF).}
\end{enumerate}

\subsection{Ma semaine de stage}
    \paragraph{J'ai commencé à apprendre à utiliser les mutex sur freeRTOS pour mettre les trois tâches ensembles et les synchroniser.Comme ça ne fonctionné pas avec seulement trois mutex (l'ordre s'effectuait correctement mais il n'y avait pas de boucle).J'ai dû utiliser une pseudo attente avec des delais assez élevé pour voir si c'était bien là le problème, en effet ce fût le cas.J'ai donc laissé de côté pour l'instant (pour demander de l'aide a Julien) et faire en attendant le git et donc rendre propre tout le code restant avec ses explications d'utilisations et les bugs eventuels. }
    
    \subsection{Les blocages}
\subsubsection{En cours}
    \paragraph{La synchronisation des tâches sous freeRTOS,l'utilisation de trois mutex pour faire un ordre entre les taches ne fonctionnent pas on doit donc utiliser un mutex mais les synchroniser avec un delay.}
\subsubsection{À venir}
    \paragraph{Aucun problème en vue pour l'instant.}
	\paragraph{}
\end{document}